\chapter{Experimental Results}
\label{chapter6}

\section{Overview} 

\section{Heterogeneous CSS Results}
\subsection{Hard Decision Combining}
\subsection{Soft Decision Combining}

\section{HST LTE-R in a Tunnel}
\subsection{K-factor in a Tunnel}
\subsection{BER Performance}
\subsection{Real-time BER in a Tunnel}

\section{Summary}

Spectral Subtraction in a extremely well studied area in signal processing, but no existing literature exists for its application in a digital communication system.  It has been shown here that it can be quite difficult for it to be applied, even under strict constraints.  Under the conditions of this thesis, the assumptions are quite reasonable, but due to the large amount of error in the results, more may need to be considered.  These may include accuracy requirements for physical equipment, primarily to reduce carrier frequency drift.  Burst scenarios may also be considered to reduce bit error rate.  Overall, for a completely non-existent field of study, these results point the possibility of operational success.  Future work will we required, especially during the implementation phase of designs.\\



