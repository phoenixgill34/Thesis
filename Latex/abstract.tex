Reliable wireless network for high speed trains requires a massive amount of data communications for enabling safety features such as train collision avoidance and railway management. Cognitive radio integrate heterogeneous wireless networks that will be deployed in order to achieve intelligent communications in future railway systems. One of the primary technical challenges in achieving a reliable communication for railways is the handling of high mobility of trains, which includes the high Doppler shifts in the transmission as well as the severe fading environment that makes it difficult to estimate wireless spectrum utilization. This thesis has two primary contributions: (1) Heterogeneous Cooperative Spectrum Sensing (CSS) prototype system, and (2) Long Term Evolution for Railways (LTE-R) performance analysis. The Heterogeneous CSS prototype system was implemented using Software-Defined Radios (SDRs) possessing different radio configurations. Both soft and hard data fusion schemes were used in order to compare the signal source detection performance in real-time fading scenarios. For future smart railways, the most effective solution is to use the underutilized spectrum as a secondary user via dynamic spectrum access (DSA). Since it is challenging to get an accurate estimate of incumbent users with a single-sensor system under a practical fading environment, the proposed cooperative spectrum sensing approach is employed instead since it can mitigate the effects of multipath and shadowing by utilizing the spatial and temporal diversity of a multiple radio network. Regarding the LTE-R contribution of this thesis, the performance analysis of high speed trains (HSTs) in tunnel environments would provide valuable insights with respect testing to the smart railway systems operating in high mobility scenarios in drastically impaired channels.


