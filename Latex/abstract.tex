Reliable wireless networks for high speed trains require a significant amount of data communications for enabling safety features such as train collision avoidance and railway management. Cognitive radio integrates heterogeneous wireless networks that will be deployed in order to achieve intelligent communications in future railway systems. One of the primary technical challenges in achieving reliable communications for railways is the handling of high mobility environments involving trains, which includes significant Doppler shifts in the transmission as well as severe fading scenarios that makes it difficult to estimate wireless spectrum utilization. This thesis has two primary contributions: (1) The creation of a Heterogeneous Cooperative Spectrum Sensing (CSS) prototype system, and (2) the derivation of a Long Term Evolution for Railways (LTE-R) system performance analysis. The Heterogeneous CSS prototype system was implemented using Software-Defined Radios (SDRs) possessing different radio configurations. Both soft- and hard-data fusion schemes were used in order to compare the signal source detection performance in real-time fading scenarios. For future smart railways, one proposed solution for enabling greater connectivity is to access underutilized spectrum as a secondary user via the dynamic spectrum access (DSA) paradigm. Since it will be challenging to obtain an accurate estimate of incumbent users via a single-sensor system within a real-world fading environment, the proposed cooperative spectrum sensing approach is employed instead since it can mitigate the effects of multipath and shadowing by utilizing the spatial and temporal diversity of a multiple radio network. Regarding the LTE-R contribution of this thesis, the performance analysis of high speed trains (HSTs) in tunnel environments would provide valuable insights with respect to the smart railway systems operating in high mobility scenarios in drastically impaired channels.
