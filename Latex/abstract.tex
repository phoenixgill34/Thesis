In recent years, the use of trains have witnessed tremendous growth due to their increasing speeds, which has led to the demand for reliable wireless communication systems with these transportation systems. The development of a reliable wireless network for high speed trains is not a simple task and it is still an emerging technology. In future trains, a massive amount of wireless communications will be used for safety features like train collision avoidance, railway management and for providing high speed internet connectivity to railway passengers. Cognitive radios will be a key technology in integrating heterogeneous wireless networks which will be deployed to achieve smart communication in future railway system. The crucial hindrance in achieving a reliable communication for railways is high mobility of trains which leads to high Doppler shifts in the transmission. The severe fading environment also makes it difficult for cognitive radios to estimate primary users to efficiently use radio spectrum.

In this thesis, we propose two test-beds namely Heterogeneous Cooperative Spectrum Sensing (CSS) and Long Term Evolution for Railways (LTE-R) performance analysis test-bed to further advance the development of smart railway communication system. Heterogeneous CSS test-bed is implemented using Software-Defined Radios (SDRs) with different radio characteristics. We used both soft and hard data fusion schemes to compare the signal source detection performance in real-time fading scenario. For future smart railways, the most effective solution is to use the underutilized spectrum as a secondary user via dynamic spectrum access (DSA). It is very challenging to get an accurate estimate of incumbent users with a single-sensor system under a practical fading environment. Various non-idealities such as shadowing, multipath and fluctuating noise variance can make it difficult to detect the primary user. Cooperative spectrum sensing can mitigate the effects of multipath and shadowing by utilizing the spatial and temporal diversity of a multiple radio network. LTE-R test-bed analyze the performance of high speed trains (HSTs) in a tunnel environment and can be used to test smart railways systems for high mobility scenario in drastically impaired channels.

