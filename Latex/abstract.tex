There have been an unprecedented increase in demand for high data rates and mobility required by new wireless applications which has led to intensive research on fifth generation (5G) wireless communication system. By 2020, 5G is projected to be employed worldwide supporting the connectivity of up to 20 billion devices and will be crucial in the success of vehicular networking and internet of things (IoT). It is also believed that 5G systems would be capable of providing significant improvements in cell capacity and will support high data rates up to 5 and 50 Gb/s for high-mobility and pedestrian users, respectively. However, to achieve such data rates for high-mobility scenarios there are still many challenges for wireless system engineers.  

In this thesis, we propose two test-beds namely Heterogeneous Cooperative Spectrum Sensing (CSS) and Long Term Evolution for Railways (LTE-R) performance analysis test-bed to further advance the 5G system development. Heterogeneous CSS test-bed is implemented using Software-Defined Radios (SDRs) with different radio characteristics. We used both soft and hard data fusion schemes to compare the signal source detection performance in real-time fading scenario. For 5G technologies, the most effective solution is to use the underutilized spectrum as a secondary user via dynamic spectrum access (DSA). It is very challenging to get an accurate estimate of incumbent users with a single-sensor system under a practical fading environment. Various non-idealities such as shadowing, multipath and fluctuating noise variance can make it difficult to detect the primary user. Cooperative spectrum sensing can mitigate the effects of multipath and shadowing by utilizing the spatial and temporal diversity of a multiple radio network. LTE-R test-bed analyze the performance of high speed trains (HSTs) in a tunnel environment and can be used to test 5G systems for high mobility scenario in drastically impaired channels.

