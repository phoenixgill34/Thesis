There has been a significant increase in the study of cognitive radios for efficiently utilizing the electromagnetic spectrum. It has been observed that the spectrum occupancy is not uniform across all frequency bands, resulting in numerous spectral white spaces. In order to more efficienty utilize the spectrum, dynamic spectrum access (DSA) has been proposed. To opportunistically access the idle channel, spectrum sensing is considered to be a significant technology enabling DSA. Although several spectrum sensing techniques have been proposed in the open literature, energy detection is widely used due to its low implementation complexity. The development of a reliable wireless network for high speed trains is not a simple task and it is still an emerging technology. Global System for Mobile Communication (GSM-R), was a wireless communications standard designed for high speed trains, but it turned out not to be reliable enough and possess several limitations. Subsequently, LTE proposed a promising solution for achieving broadband data rates in high speed trains that can overcome various GSM-R limitations.
